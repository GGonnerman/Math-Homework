\documentclass{article}

\usepackage{amsmath, amssymb} % Math formatting and symbols
\usepackage{graphicx} % Insert graphics
\usepackage{wrapfig} % Allow text to wrap around images
\usepackage[cm]{fullpage} % Smaller margins and header/footer
\usepackage{setspace}
\usepackage{gensymb} % degree symbol
\usepackage{amstext} % for \text macro
\usepackage{array}   % for \newcolumntype macro
\usepackage{float}

\usepackage{enumitem} % Allow for widest tag in enumerate

\usepackage{etoolbox}
\newcommand{\zerodisplayskips}{%
	\setlength{\abovedisplayskip}{0pt}%
	\setlength{\belowdisplayskip}{0pt}%
	\setlength{\abovedisplayshortskip}{0pt}%
	\setlength{\belowdisplayshortskip}{0pt}}
\appto{\normalsize}{\zerodisplayskips}
\appto{\small}{\zerodisplayskips}
\appto{\footnotesize}{\zerodisplayskips}

\renewcommand*\descriptionlabel[1]{\hspace\leftmargin$#1$}

\newenvironment{adescription}[1]
{\begin{list}{}%
		{\renewcommand\makelabel[1]{##1\hfill}%
			\settowidth\labelwidth{\makelabel{#1}}%
			\setlength\leftmargin{\labelwidth}
			\addtolength\leftmargin{\labelsep}}}
	{\end{list}}

\newcommand{\bd}{\textbf}
\newcommand{\dquad}{\quad{}\quad{}}
\newcommand{\nine}{\frac{\pi}{2}}
\newcommand{\dx}{\frac{d}{dx}}
\newcommand{\nh}{\\ \hline}

\newcolumntype{L}{>{$}l<{$}}

\title{\vspace{-5ex}Derivatives Cheat sheet  \vspace{-5ex}}
\author{}
\date{}

\begin{document}
	\setstretch{1}
	\maketitle{}
	
	\subsection*{How to write}
	Rect $(x, y)$ \\
	Polar $(r, \theta)$ \\
	Compass: $r \text{ at } \theta$
	
	\subsection*{Polar Coordinates}
	
	\begin{align*}
		\cos(\theta) &= \frac{x}{r} && & \sin(\theta) &= \frac{y}{r} \\
		x &= r * \cos(\theta) && & y &= r * \sin(\theta) \\
		r^2 &= \pm \sqrt{x^2 + y^2} & & & \theta &= \tan^{-1}(\frac{y}{x})
	\end{align*}

	\noindent $\theta$ is counter-clockwise from the positive x-axis \\
	$\theta$ may need $\pm$ 180 or $\pi$  \\
	Polar things can be represented many different ways
	
	\subsection*{Compass Coordinates}
	$\theta$ is clockwise from the positive y-axis and is always positive \\
	r and $\theta$ are both always positive \\
	\emph{visualize where it should be}
	
	\subsection*{Imaginary Numbers}
	$a+bi$ $\rightarrow$ $r(\cos(\theta)+i\sin(\theta)) = r*cis(\theta)$ \\
	\indent $r = sqrt(a^2+b^2)$ \\
	\indent $\theta = \tan^{-1}(\frac{b}{a})$ \\
	\indent \indent Make sure in right quadrant
	
	\noindent Multiplying
	\begin{align*}
		[r_1 cis(\theta_1)] * [r_2 cis(\theta_2)] = (r_1*r_2) cis(\theta_1+\theta_2)
	\end{align*}
	Dividing
	\begin{align*}
		\frac{r_1 cis(\theta_1)}{ r_2 cis(\theta_2)} = (\frac{r_1}{r_2}) cis(\theta_1-\theta_2)
	\end{align*}

	\noindent Raising to a power (n $\ge$ 1 and positive integer)
	\begin{align*}
		[r cis(\theta)]^n = r^n cis(n * \theta)
	\end{align*}

	\noindent Getting $n^\text{th}$ root of 
	\begin{align*}
		[r cis(\theta)]^\frac{1}{n} = r^\frac{1}{n} cis(\frac{\theta}{n} + \frac{360}{n} * k)
	\end{align*}
	where k = 0, 1, 2, ..., n - 1
	
	

\end{document}