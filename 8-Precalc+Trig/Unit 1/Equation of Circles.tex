\documentclass{article}

\usepackage{amsmath, amssymb} % Math formatting and symbols
\usepackage{graphicx} % Insert graphics
\usepackage{wrapfig} % Allow text to wrap around images
\usepackage[cm]{fullpage} % Smaller margins and header/footer
\usepackage{setspace}
\usepackage{gensymb} % degree symbol

\usepackage{enumitem} % Allow for widest tag in enumerate

\renewcommand*\descriptionlabel[1]{\hspace\leftmargin$#1$}

\newenvironment{adescription}[1]
{\begin{list}{}%
	{\renewcommand\makelabel[1]{##1\hfill}%
		\settowidth\labelwidth{\makelabel{#1}}%
		\setlength\leftmargin{\labelwidth}
		\addtolength\leftmargin{\labelsep}}}
{\end{list}}

\newcommand{\bd}{\textbf}
\newcommand{\dquad}{\quad{}\quad{}}

\title{Graph of Equations}
\author{}
\date{}

\begin{document}
	\setstretch{0.5}
	\maketitle{}
	
	\section{Review}
		Assume...
		\begin{adescription}{\dquad{}$P_2$:}
			\item[\dquad{}$P_1$:] $(x_1, y_1)$
			\item[\dquad{}$P_2$:] $(x_2, y_2)$
		\end{adescription}
%		\begin{adescription}{\dquad{}$(x_2, y_2)$:}
%			\item[\dquad{}$(x_1, y_1)$:] Point 1
%			\item[\dquad{}$(x_2, y_2)$:] Point 2
%		\end{adescription}
		\subsection{Distance Formula}
			\begin{equation}
				d = \sqrt{(x_2 - x_1)^2 + (y_2 - y_1)^2}
			\end{equation}
			where:
			\begin{adescription}{\dquad{}$d$:}
				\item[\dquad{}$d$:] Distance between $P_1$ and $P_2$
			\end{adescription}
	
		\subsection{The Midpoint Formula}
			\begin{equation}
				m = (\frac{x_1+x_2}{2}, \frac{y_1+y_2}{2})
			\end{equation}
			where:
			\begin{adescription}{\dquad{}$m$:}
				\item[\dquad{}$m$:] Midpoint between $P_1$ and $P_2$
			\end{adescription}
	
	\section{Equations of Circles}
		You can draw a circle using an \bd{relationship} not a function.
		\begin{equation}
			(x - h)^2 + (y - k)^2 = r^2
		\end{equation}
		where:
			\begin{adescription}{\dquad{}$(h, k)$:}
				\item[\dquad{}$(h, k)$:] Center Point
				\item[\dquad{}$r$:] Radius
			\end{adescription}
	
	\section{Symmetry}
		\subsection{Y-Axis}
			\begin{itemize}[label=\dquad{}--]
				\item Called an "\bd{Even Function}"
				\item Looks the same after reflection over Y-Axis
				\item Has to meet the following requirement(s)...
			\end{itemize}
			\begin{equation}
				f(x) = f(-x)
			\end{equation}
			\par One example of such a function is $y = x^2$.
			\begin{align*}
				f(4) &= 16 \\
				f(-4) &= 16 \\
				16 &= 16
			\end{align*}
		
		\subsection{X-Axis}
			\begin{itemize}[label=\dquad{}--]
				\item \bd{Not a function}, doesn't pass vertical line test
				\item Called a \bd{relationship}
				\item Has to meet the following requirement(s)...
			\end{itemize}
			\begin{equation}
				x \mapsto \{-y,  y\}
			\end{equation}
			One example of such a equation is $x = y^2$ but \bd{not} $y = \sqrt{x}$ because that would only allow positive x values.
			\begin{align*}
				9^2 &= 81 \\
				(-9)^2 &= 81
			\end{align*}
		
		\subsection{Origin}
			\begin{itemize}[label=\dquad{}--]
				\item Called an "\bd{Odd Function}"
				\item Visually the same after $180\degree{}$ rotation about $(0, 0)$
				\item Has to meet the following requirement(s)...
			\end{itemize}
			\begin{align}
				f(x) &= y \\
				f(-x) &= -y
			\end{align}
		One example of such a function is $y = x^3$
		\begin{align*}
			f(2) &= 8 \\
			f(-2) &= -8
		\end{align*}

\end{document}