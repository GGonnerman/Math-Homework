\documentclass[twocolumn]{article}

\usepackage{amsmath, amssymb} % Math formatting and symbols
\usepackage{graphicx} % Insert graphics
\usepackage{wrapfig} % Allow text to wrap around images
\usepackage[cm]{fullpage} % Smaller margins and header/footer
\usepackage{setspace}
\usepackage{gensymb} % degree symbol

\usepackage{enumitem} % Allow for widest tag in enumerate

\renewcommand*\descriptionlabel[1]{\hspace\leftmargin$#1$}

\newenvironment{adescription}[1]
{\begin{list}{}%
	{\renewcommand\makelabel[1]{##1\hfill}%
		\settowidth\labelwidth{\makelabel{#1}}%
		\setlength\leftmargin{\labelwidth}
		\addtolength\leftmargin{\labelsep}}}
{\end{list}}

\newcommand{\bd}{\textbf}
\newcommand{\dquad}{\quad{}\quad{}}

\title{Graph of Equations}
\author{}
\date{}

\begin{document}
	\setstretch{0.5}
	\maketitle{}
	
	\section{Review}
		Assume...
		\begin{adescription}{\dquad{}$P_2$:}
			\item[\dquad{}$P_1$:] $(x_1, y_1)$
			\item[\dquad{}$P_2$:] $(x_2, y_2)$
		\end{adescription}
%		\begin{adescription}{\dquad{}$(x_2, y_2)$:}
%			\item[\dquad{}$(x_1, y_1)$:] Point 1
%			\item[\dquad{}$(x_2, y_2)$:] Point 2
%		\end{adescription}
		\subsection{Distance Formula}
			\begin{equation*}
				d = \sqrt{(x_2 - x_1)^2 + (y_2 - y_1)^2}
			\end{equation*}
			where:
			\begin{adescription}{\dquad{}$d$:}
				\item[\dquad{}$d$:] Distance between $P_1$ and $P_2$
			\end{adescription}
	
		\subsection{The Midpoint Formula}
			\begin{equation*}
				m = (\frac{x_1+x_2}{2}, \frac{y_1+y_2}{2})
			\end{equation*}
			where:
			\begin{adescription}{\dquad{}$m$:}
				\item[\dquad{}$m$:] Midpoint between $P_1$ and $P_2$
			\end{adescription}
	
	\section{Equations of Circles}
		You can draw a circle using an \bd{relationship} not a function.
		\begin{equation*}
			(x - h)^2 + (y - k)^2 = r^2
		\end{equation*}
		where:
			\begin{adescription}{\dquad{}$(h, k)$:}
				\item[\dquad{}$(h, k)$:] Center Point
				\item[\dquad{}$r$:] Radius
			\end{adescription}
	
	\section{Symmetry}
		\subsection{Y-Axis}
			\begin{itemize}[label=\dquad{}--]
				\item Called an "\bd{Even Function}"
				\item Looks the same after reflection over Y-Axis
				\item Has to meet the following requirement(s)...
			\end{itemize}
			\begin{equation*}
				f(x) = f(-x)
			\end{equation*}
			\par One example of such a function is $y = x^2$.
			\begin{align*}
				f(4) &= 16 \\
				f(-4) &= 16 \\
				16 &= 16
			\end{align*}
		
		\subsection{X-Axis}
			\begin{itemize}[label=\dquad{}--]
				\item \bd{Not a function}, doesn't pass vertical line test
				\item Called a \bd{relationship}
				\item Has to meet the following requirement(s)...
			\end{itemize}
			\begin{equation*}
				x \mapsto \{-y,  y\}
			\end{equation*}
			One example of such a equation is $x = y^2$ but \bd{not} $y = \sqrt{x}$ because that would only allow positive x values.
			\begin{align*}
				9^2 &= 81 \\
				(-9)^2 &= 81
			\end{align*}
		
		\subsection{Origin}
			\begin{itemize}[label=\dquad{}--]
				\item Called an "\bd{Odd Function}"
				\item Visually the same after $180\degree{}$ rotation about $(0, 0)$
				\item Has to meet the following requirement(s)...
			\end{itemize}
			\begin{align*}
				f(x) &= y \\
				f(-x) &= -y
			\end{align*}
		One example of such a function is $y = x^3$
		\begin{align*}
			f(2) &= 8 \\
			f(-2) &= -8
		\end{align*}
	
	\section{Equations of  Lines}
	Assume...
	\begin{adescription}{\dquad{}$P_2$:}
		\item[\dquad{}$m$:] Slope
	\end{adescription}
	  \subsection{Slope}
	  	\begin{equation*}
	  		m = \frac{\text{"Ryse"}}{\text{Run}} = \frac{y_2-y_1}{x_2-x_1}
	  	\end{equation*}
	  \subsection{Forms}
	    \subsubsection{Slope-Intercept Form}
	    	\begin{equation*}
	    		y = mx + b
	    	\end{equation*}
    		where:
    		\begin{adescription}{\dquad{}$b$:}
    			\item[\dquad{}$b$:] x-intercept
    		\end{adescription}
	    \subsubsection{Point Slope Form}
	    	If you need point-slope form, just sub out values. However, if you need to find slope-intercept form you can solve for $y$.
	    	\begin{equation*}
	    		y - y_1 = m(x - x_1)
	    	\end{equation*}
	    \subsubsection{Intercept Form}
	    	\begin{equation*}
	    		\frac{x}{a} + \frac{y}{b} = 1
	    	\end{equation*}
    		where:
    		\begin{adescription}{\dquad{}$b$:}
    			\item[\dquad{}$a$:] x-intercept, point (a, 0) falls on the line
    			\item[\dquad{}$b$:] y-intercept, point (0, b) falls on the line
    		\end{adescription}
    		This form can be converted into \bd{General Form} through the multiplication of the least common multiple of $a$ and $b$. Then subtracting   the value on the right side of the equation.
	    \subsubsection{General Form}
	    	\begin{equation*}
	    		Ax + By + C = 0
	    	\end{equation*}
    		where:
    		\begin{itemize}[label=--]
    			\item[] $A$ is non-negative
    			\item[] $A$, $B$, and $C$ are all \emph{integers}
    		\end{itemize}
	  \subsection{Relationships of Lines}
	    \subsubsection{Parallel Lines}
	    	-- \bd{same slopes}.
	    \subsubsection{Perpendicular Lines}
	    	-- \bd{opposite reciprocal slopes}. \\
	    	Consider the following where lines $t_1$ and $t_2$ are perpendicular.
	    	\begin{align*}
	    		t_1 &= 3/8 \\
	    		t_2 &= -8/3
	    	\end{align*}
    
    \section{Functions and Equations}
    
    \subsection{Is it a function?}
   	-- Each $x$ only maps to one $y$
   	
   	\section{Domain \& Range}
   	
   	\subsection{Formatting}
   	Example...
   	\begin{align*}
   		D&: (-1,2] \\
   		R&: (-\infty, 12)
   	\end{align*}
   \begin{itemize}[label=--]
   	\item "(", ")" means exclusive
   	\item "[", "]" means inclusive
   	\item \bd{Never} use [] with $\infty$
   \end{itemize}
   	
   	\subsection{Zeros}
   	Solve for when $y = 0$ \\
   	They are x-intercepts
   	
   	\subsection{Increasing and Decreasing}
   	\par \bd{Never} use "[", "]", always "(", ")" \\
   	\bd{Always} Least $\to$ Greatest \\
   	
   	
   	\subsection{Relative Maximum and Minimum}
   	A \bd{point} on a line where the line is either above on both sides (\emph{Minimum}) or below on both sides (\emph{Maximum}). \bd{Cannot} be an \bd{end point}.
   	
   	
   	\subsection{New Functions}
   	
   	\subsubsection{Greatest Integer Function (Floor)}
    Represented by
    \begin{equation*}
    	f(x) = [[x]]
    \end{equation*}
   	
   	Left side solid (\emph{Included}), right side empty (\emph{Excluded})
   	
   	\subsubsection{Peace Function}
   	\par An equation, but with conditionals
   	
   	Example...
   	\begin{align*}
	   	f(x) = 
	   	\begin{cases}
	   		x^2 - 3      &\quad \text{if } x \ge 3 \\
	   		-2x^4 + 9x^3 &\quad \text{if } x < 3
	   	\end{cases}
   \end{align*}
 
 		Plug it into calculator by multiplying things and conditions
 		
 		\begin{equation*}
 			f(x) = (x^2 - 3)(x \ge 3) + (-2x^4 + 9x^3)(x < 3)
 		\end{equation*}

\end{document}