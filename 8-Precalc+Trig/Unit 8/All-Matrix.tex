\documentclass{article}

\usepackage{amsmath, amssymb} % Math formatting and symbols
\usepackage{graphicx} % Insert graphics
\usepackage{wrapfig} % Allow text to wrap around images
\usepackage[cm]{fullpage} % Smaller margins and header/footer
\usepackage{setspace}
\usepackage{gensymb} % degree symbol
\usepackage{amstext} % for \text macro
\usepackage{array}   % for \newcolumntype macro
\usepackage{float}

\usepackage{enumitem} % Allow for widest tag in enumerate

\usepackage{etoolbox}
\newcommand{\zerodisplayskips}{%
	\setlength{\abovedisplayskip}{0pt}%
	\setlength{\belowdisplayskip}{0pt}%
	\setlength{\abovedisplayshortskip}{0pt}%
	\setlength{\belowdisplayshortskip}{0pt}}
\appto{\normalsize}{\zerodisplayskips}
\appto{\small}{\zerodisplayskips}
\appto{\footnotesize}{\zerodisplayskips}

\renewcommand*\descriptionlabel[1]{\hspace\leftmargin$#1$}

\newenvironment{adescription}[1]
{\begin{list}{}%
		{\renewcommand\makelabel[1]{##1\hfill}%
			\settowidth\labelwidth{\makelabel{#1}}%
			\setlength\leftmargin{\labelwidth}
			\addtolength\leftmargin{\labelsep}}}
	{\end{list}}

\newcommand{\bd}{\textbf}
\newcommand{\dquad}{\quad{}\quad{}}
\newcommand{\nine}{\frac{\pi}{2}}
\newcommand{\dx}{\frac{d}{dx}}
\newcommand{\nh}{\\ \hline}

\newcolumntype{L}{>{$}l<{$}}

\title{\vspace{-5ex}Matricies Cheat sheet  \vspace{-5ex}}
\author{}
\date{}

\begin{document}
	\setstretch{1}
	\maketitle{}

	\subsection*{Matricies that will be used}
	\begin{equation*}
		A = 
		\begin{bmatrix}
			a & b \\
			c & d
		\end{bmatrix}
		\text{ }
		B = 
		\begin{bmatrix}
			e & f \\
			g & h
		\end{bmatrix}
	\end{equation*}

	\subsection*{Basics}
	Naming: Matrix is row by column \\
	Indexing: $C = \begin{bmatrix}
			c_{1, 1} & c_{1, 2} \\
			c_{2, 1} & c_{2, 2}
		\end{bmatrix}$ \\
	Multiplication (only possible if A column count = B row count): $AB = 		\begin{bmatrix}
		ae+bg & af+bh \\
		ce+dg & cf+dh
	\end{bmatrix}$ \\
	Determinant of 2x2: $\det(A) = ad-bc$ \\
	Inverse of 2x2: $A^{-1} = \frac{1}{\det(A)} \begin{bmatrix}
		d & -b \\
		-c & a
	\end{bmatrix}$

	\subsection*{Solving Equations}
	A = coefficient matrix \\
	B = constant matrix
	\subsubsection*{Inverse Matrix}
	$A^{-1}B= \begin{bmatrix}
		x \\
		y
	\end{bmatrix}$
	\subsubsection*{Gaussian Elimination}
	Augment A and B, then manipulate so A looks like identity matrix \\
	\indent 0 0 ... 0 :  0 = infinity \\
	\indent 0 0 ... 0 : n = no solution \\
	Example for infinity solution: $\begin{bmatrix}
		1 & 0 & 2 & 7 \\
		0 & 1 & -3 & 4
	\end{bmatrix}$ \\
	\indent $x + 2z = 7; x = 7 - 2z$ \\
	\indent $y - 3z = 4; y = 4 + 3z$ \\
	\indent For some reason(?) $z = a$ \\
	\indent So solution, $(7 - 2a, 4+3a, a)$
	
	\subsubsection*{Cramers Rule}
	$X = \frac{1}{\det(A)}\begin{bmatrix}
		\text{Coeff matrix w/ X column swapped for constant}
	\end{bmatrix}$ \\
	Same for all other variables \\
	Special Cases: \\
	\indent $\frac{0}{0} = \text{infinity solutions}$ \\
	\indent $\frac{n}{0} = \text{no solutions}$

\end{document}