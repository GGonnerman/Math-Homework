\documentclass[]{article}

\usepackage{amsmath, amssymb} % Math formatting and symbols
\usepackage{graphicx} % Insert graphics
\usepackage{wrapfig} % Allow text to wrap around images
\usepackage[cm]{fullpage} % Smaller margins and header/footer
\usepackage{setspace}
\usepackage{gensymb} % degree symbol

\usepackage{enumitem} % Allow for widest tag in enumerate

\renewcommand*\descriptionlabel[1]{\hspace\leftmargin$#1$}

\newenvironment{adescription}[1]
{\begin{list}{}%
		{\renewcommand\makelabel[1]{##1\hfill}%
			\settowidth\labelwidth{\makelabel{#1}}%
			\setlength\leftmargin{\labelwidth}
			\addtolength\leftmargin{\labelsep}}}
	{\end{list}}

\newcommand{\bd}{\textbf}
\newcommand{\dquad}{\quad{}\quad{}}

\title{Graph of Equations}
\author{}
\date{}

\begin{document}
	%\setstretch{0.5}
	\maketitle{}

	\section{Trig function transformations}
	Assume...
	\begin{adescription}{\dquad{}$d$:}
		\item[\dquad{}$a$:] Vertical stretch/shrink
		\item[\dquad{}$b$:] Horizontal stretch/shrink
		\item[\dquad{}$c$:] Horizontal "phase" shift
		\item[\dquad{}$d$:] Vertical shift
	\end{adescription}
	\begin{equation}
		y = d + a * sin(bx - c)
	\end{equation}
	Example...
	\par A sine curve with a period $\pi$, amplitude of 2, right phrase of $\frac{\pi}{2}$, and a vertical shift down 4 \\
	$a$ is derived from amplitude, positive if reflection is not specified, $a = 2$  \\
	$d$ is derived from vertical shift, $d = -4$
	\begin{equation*}
		b = \frac{2\pi}{\text{period}} = \frac{2\pi}{\pi} = 2
	\end{equation*}
	\begin{equation*}
		c = b * \text{shift} = 2 * \frac{\pi}{2} = \pi
	\end{equation*}
	Then you put it all together...
	\begin{equation*}
		y = -4 + 2 * \sin(2x - \pi)
	\end{equation*}
	
	Pure equations...
	\begin{align*}
		a &= \text{amplitude} \\
		b &= \frac{2\pi}{\text{period}} \\
		c &= 2 * \text{phase shift} \\
		d &= \text{vert shift}
	\end{align*}

\end{document}