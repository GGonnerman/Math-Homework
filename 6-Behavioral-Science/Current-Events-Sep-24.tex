\documentclass{article}

\usepackage{fullpage}
\usepackage{enumitem}

\title{Current Events}
\author{Gaston Gonnerman}
\date{6th period}

\begin{document}
	\maketitle
	
	\begin{center}
		\large{
			\textbf{Bus-size asteroid to zoom by Earth, ducking below satellites}
		}
	\end{center}
	\par This coming Thursday, September 24th, it is expected that a fairly small asteroid will be very close to earth. It is sized about the size of a school bus, 15-30 feet in diameter, and will be approximately 13,000 miles from Earth--significantly closer than the moon and many satellites. After passing by the Earth, it won't return back to Earth's neighborhood until 2041. 
	
	\begin{itemize}[label=--]
		\item Approximately how large is the asteroid? \textbf{The size of a school bus}
		\item Is the asteroid expected to be closer to Earth than the moon is? \textbf{Yes}
	\end{itemize}

	\noindent
	\emph{Source:} https://apnews.com/b7c9b47634768b0a68f7f7e1e7821d28
	\\ \\ \\
	
	\begin{center}
		\large{
			\textbf{California is ready to pull the plug on gas vehicles}
		}
	\end{center}
	\par Last Wednesday, California's governor, Gavin Newsom, announced a new plan regarding the sale of automobiles. He stated that the sale of gasoline-powered passenger cars and trucks would be banned in approximately 15 years (2035). This serves to decrease emissions while also boosting the electric-based car manufacturers based on California--namely Tesla.
	
	\begin{itemize}[label=--]
		\item The sale of what is expected to ban the sale of in 15 years in California? \textbf{Gas-based cars and trucks}
		\item Which state plans to ban the sale of gas-based cars in 2035? \textbf{California}
	\end{itemize}

	\noindent
	\emph{Source:} https://apnews.com/4956d87b72b000a917eed27392d16d8b
\end{document}