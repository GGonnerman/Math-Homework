\documentclass{article}

\usepackage{fullpage}
\usepackage{enumitem}

\title{Current Events}
\author{Gaston Gonnerman}
\date{6th period}

\begin{document}
	\maketitle
	
	\begin{center}
		\large{
			\textbf{Astronomers may have found a signature of life on Venus}
		}
	\end{center}
	\par Recently, there was a gas known as \emph{phosphine} found on the planet Venus by a group of scientists at MIT. This may not seem important; however, given current scientific knowledge, phosphine is not known to occur without some form of life. This leads to the two conclusions that either some form of, likely aerial, life or there is some chemical process that has yet to be discovered by scientists.
	
	\begin{itemize}[label=--]
		\item What planet was potential evidence of life recently found on? \textbf{Venus}
		\item Where did the scientists work who discovered the phosphine gas on Venus? \textbf{MIT}
	\end{itemize}

	\noindent
	\emph{Source:} https://news.mit.edu/2020/life-venus-phosphine-0914
	\\ \\ \\
	
	\begin{center}
		\large{
			\textbf{The US just charged a group of Chinese hackers with orchestrating 'unprecedented' cyberattacks targeting over 100 companies, government agencies, and universities}
		}
	\end{center}
	\par Unsealed Wednesday, reports show federal prosecutors accusing China and Malaysia with cyberattacks on over 100 companies, agencies, and nonprofits worldwide. These hacks were said to be part of a hacking operation called APT41. Though no specifics were given about who was on the receiving end of the attack, large names such as Microsoft, Facebook, Apple, Google, and Verizon were notes as having assisted the government in the investigation.
	
	\begin{itemize}[label=--]
		\item What hacking operation were these hackers likely a part of? \textbf{APT41}
		\item What region were these hackers primarily based in? \textbf{China}
	\end{itemize}

	\noindent
	\emph{Source:} https://www.businessinsider.com/chinese-hackers-target-over-100-companies-with-cyberattacks-doj-2020-9
\end{document}