\documentclass{article}

\usepackage{fullpage}
\usepackage{enumitem}

\title{Current Events}
\author{Gaston Gonnerman}
\date{6th period}

\begin{document}
	\maketitle
	
	\begin{center}
		\large{
			\textbf{Building wave of ransomware attacks strike U.S. hospitals}
		}
	\end{center}
	\par Since the beginning of the Covid-19 pandemic, there has been a large increase in digital attacks. This has now extended to hospitals with Eastern European criminals having targeted over a dozen hospitals. Currently, these attacked are being investigated by the FBI in hopes to decrease or stop these attacks in the future. After getting ransomware, some hospitals have been forced into using only paper, which decrease their ability to transfer patients or function at a very high level.
	
	\begin{itemize}[label=--]
		\item What industry has been getting attacked by ransomware recently? \textbf{Hospitals}
		\item Some hospitals have been forced to using only paper -- why? \textbf{Ransomware Attacks}
	\end{itemize}

	\noindent
	\emph{Source:} https://www.reuters.com/article/us-usa-healthcare-cyber-idUSKBN27D35U
	\\ \\ \\
	
	\begin{center}
		\large{
			\textbf{Antarctica yields oldest fossils of giant birds with 21-foot wingspans}
		}
	\end{center}
	\par Recently unearthed, the pelagornithid is a  giant, ancient bird having lived over 60 million years ago. This bird had a toothed beak and an apparent wingspan stretched 21 feet, almost twice the largest wingspan of a bird currently alive. These findings have helped to put into proportion the massive difference in size between some ancient creatures and those of today.
	
	\begin{itemize}[label=--]
		\item How many feet was this pelagornithid's wingspan? \textbf{21 feet}
		\item The ancient pelagornithid had what on its beak? \textbf{Teeth}
	\end{itemize}

	\noindent
	\emph{Source:} https://news.berkeley.edu/2020/10/27/antarctica-yields-oldest-fossils-of-giant-birds-with-21-foot-wingspans/
\end{document}
