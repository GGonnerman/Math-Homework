\documentclass{article}

\usepackage{fullpage}
\usepackage{enumitem}

\title{Current Events}
\author{Gaston Gonnerman}
\date{6th period}

\begin{document}
	\maketitle
	
	\begin{center}
		\large{
			\textbf{US gives first-ever OK for small commercial nuclear reactor}
		}
	\end{center}
	\par A design for a nuclear reactor was recently put forth for approval and later approved by U.S. officials for the first time. This reactor would be able to provide power for more than 50,000 homes and would have a smaller carbon footprint and be safer than many older power reactors. There is currently 1 scheduled to be built in 2029, followed by 11 more in 2030, assuming everything goes to plan.
	
	\begin{itemize}[label=--]
		\item How many homes would a single nuclear reactor be able to provide power for? \textbf{More than 50,000 homes}
		\item How many of these commercial nuclear reactors are expected to be built total? \textbf{12 reactors}
	\end{itemize}

	\noindent
	\emph{Source:} https://apnews.com/910766c07afd96fbe2bd875e16087464
	\\ \\ \\
	
	\begin{center}
		\large{
			\textbf{Amazon wins FAA approval to deliver packages by drone}
		}
	\end{center}
	\par Since as early as 2013, Amazon has had its eyes on drone-based delivery but continually was slowed and by regulations. Recently, however, Amazon gained approval from the FAA to deliver packages by drone, becoming the third company to have approval after UPS and Google. The models revealed last year were self-piloting, fully electric, able to carry up to 5 pounds, and deliver good in 30 minutes by dropping them in a backyard so these are likely similar to what will be used in testing.
	
	\begin{itemize}[label=--]
		\item How many companies have drone flight approval from the FAA? \textbf{Three companies}
		\item How many pounds were their test drone able to lift (last year)? \textbf{5 pounds}
	\end{itemize}

	\noindent
	\emph{Source:} https://www.msn.com/en-us/finance/companies/amazon-wins-faa-approval-to-deliver-packages-by-drone/ar-BB18yLXZ
\end{document}