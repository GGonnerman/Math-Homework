\documentclass[twocolumn]{article}

\usepackage{amsmath, amssymb} % Math formatting and symbols
\usepackage{graphicx} % Insert graphics
\usepackage{wrapfig} % Allow text to wrap around images
\usepackage[cm]{fullpage} % Smaller margins and header/footer
\usepackage{setspace}

\usepackage{enumitem} % Allow for widest tag in enumerate

\renewcommand*\descriptionlabel[1]{\hspace\leftmargin$#1$}

\newenvironment{adescription}[1]
{\begin{list}{}%
	{\renewcommand\makelabel[1]{##1\hfill}%
		\settowidth\labelwidth{\makelabel{#1}}%
		\setlength\leftmargin{\labelwidth}
		\addtolength\leftmargin{\labelsep}}}
{\end{list}}

\newcommand{\bd}{\textbf}
\newcommand{\dquad}{\quad{}\quad{}}

\title{Interest}
\author{}
\date{}

\begin{document}
	\setstretch{0.5}
	\maketitle{}

	\subsection*{1 Preface}
	\quad{}For the remainder of this paper, the following variables will be as set forth, unless specified otherwise.
	
	\begin{adescription}{\quad{}$m$:}
		\item[\dquad{}$A$:] Accumulated Amount (\emph{Future Value})
		\item[\dquad{}$P$:] Principal (\emph{Present Value})
		\item[\dquad{}$r$:] Nominal Interest Rate Per Year
		\item[\dquad{}$m$:] Yearly Number of Conversion Periods
		\item[\dquad{}$t$:] Term (Number of Years)
		\item[\quad{}As well as...]
		\item[\dquad{}$i$:] Interest Rate Per Period
		\begin{equation}
			\frac{r}{m}
		\end{equation}
		\item[\dquad{}$n$:] Total Number of Conversion Periods
		\begin{equation}
			m * t
		\end{equation}
	\end{adescription}
	\setstretch{1}
	
	\subsection*{2 Simple Interest}
	
	\begin{equation}
		A = P(1 + rt)
	\end{equation}
	
	\subsection*{3 Compound Interest} 
	\begin{itemize}[label=--]
		\item Interest that is periodically added to the principal
		\item Earns interest on itself
	\end{itemize}
	
	\begin{equation}
		A = P(1 + i)^n
	\end{equation}
	
	\subsection*{4 Continuous Compounding Interest}

	\begin{equation}
		A = Pe^{rt}
	\end{equation}
	
	\subsection*{5 Effective Rate of Interest}
	\dquad{}The \bd{effective rate of interest} is the \bd{annual rate} which would yield the \bd{same accumulated amount} as the \bd{nominal rate} ($r$) compounded $m$ times over the term ($t$). It can also be called the \bd{annual percentage yield}.
	\begin{equation}
		r_{eff} = (1 + i)^m - 1
	\end{equation}
	where:
	\begin{adescription}{\quad{}$r_{eff}$:}
		\item[\quad{}$r_{eff}$:] Effective Rate of Interest
	\end{adescription}
	
	\subsection*{6 Present Value}

	    \subsubsection*{\dquad{}6.1 Compound Interest}
	    
	    \begin{equation}
	    	P = A(1 + i)^{-n}
	    \end{equation}
	    
	    \subsubsection*{\dquad{}6.2 Continuous Interest}
	 
	    \begin{equation}
	    	P = Ae^{-rt}
	    \end{equation}

\end{document}