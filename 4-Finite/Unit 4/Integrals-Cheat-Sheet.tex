\documentclass{article}

\usepackage{amsmath, amssymb} % Math formatting and symbols
\usepackage{graphicx} % Insert graphics
\usepackage{wrapfig} % Allow text to wrap around images
\usepackage[cm]{fullpage} % Smaller margins and header/footer
\usepackage{setspace}
\usepackage{gensymb} % degree symbol
\usepackage{amstext} % for \text macro
\usepackage{array}   % for \newcolumntype macro
\usepackage{float}

\usepackage{enumitem} % Allow for widest tag in enumerate

\usepackage{etoolbox}
\newcommand{\zerodisplayskips}{%
	\setlength{\abovedisplayskip}{0pt}%
	\setlength{\belowdisplayskip}{0pt}%
	\setlength{\abovedisplayshortskip}{0pt}%
	\setlength{\belowdisplayshortskip}{0pt}}
\appto{\normalsize}{\zerodisplayskips}
\appto{\small}{\zerodisplayskips}
\appto{\footnotesize}{\zerodisplayskips}

\renewcommand*\descriptionlabel[1]{\hspace\leftmargin$#1$}

\newenvironment{adescription}[1]
{\begin{list}{}%
		{\renewcommand\makelabel[1]{##1\hfill}%
			\settowidth\labelwidth{\makelabel{#1}}%
			\setlength\leftmargin{\labelwidth}
			\addtolength\leftmargin{\labelsep}}}
	{\end{list}}

\newcommand{\bd}{\textbf}
\newcommand{\dquad}{\quad{}\quad{}}
\newcommand{\nine}{\frac{\pi}{2}}
\newcommand{\dx}{\,dx}
\newcommand{\nh}{\\ \hline}

\newcolumntype{L}{>{$}l<{$}}

\title{\vspace{-5ex}Integral Cheat sheet  \vspace{-5ex}}
\author{}
\date{}

\begin{document}
	\setstretch{1}
	\maketitle{}
	
	\subsection*{Common Functions}
	\begin{center}
	\begin{tabular}{| L | L |}
		\hline
		\text{Equation} & \text{Antiderivative}
		\nh 0 & C
		\nh k & kx + C
		\nh kf(x) \dx & k \int f(x) \dx
		\nh \int [f(x) \pm g(x)] \dx & \int f(x) \dx \pm \int g(x) \dx
		\nh e^x & e^x + C
		\nh a^x & (\frac{1}{\ln(a)})a^x+C
		\nh \frac{1}{x} & \ln|x| + C
		\nh \ln{x} & x(\ln|x| - 1) + C
		\nh \log_b(x) & \frac{1}{\ln(b)} [x(\ln|x|-1)] + C
		\nh
	\end{tabular}
\end{center}

\subsection*{Trigonometric Integrals}
\begin{align*}
	& \int sin = -cos + C & & \int cos = sin + C \\
	& \int tan = -\ln|cos| + C & & \int cot = \ln|sin| + C \\
	& \int sec = \ln|sec + tan| + C & & \int csc = \ln|csc - cot| + C
\end{align*}

\subsubsection*{Abnormal Trigonometric Integrals}
\begin{align*}
	& \int sec^2 = tan + C & & \int sec*tan = sec + C \\
	& \int csc^2 = -cot + C & & \int csc*cot = -csc + C \\
	& \int tan = \ln|sec| & &
\end{align*}

\subsubsection*{Inverse Trigonometric Integrals (a is positive)}
\begin{align*}
	& \int \frac{1}{ \sqrt{a^2 - x^2} } = \arcsin(\frac{x}{a}) + C \\
	& \int \frac{1}{a^2 + x^2} = \frac{1}{a} \arctan(\frac{x}{a}) + C \\
	& \int \frac{1}{x \sqrt{x^2-a^2} } = \frac{1}{a} \text{arcsec}(\frac{|x|}{a}) + C
\end{align*}

\subsection*{Integration Rules}
\indent \indent Power Rule ($n \ne -1$)
\begin{equation*}
	\int x^n = \frac{x^{n+1}}{n+1} + C
\end{equation*}

\end{document}
